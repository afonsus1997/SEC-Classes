\section{Introduction}

\noindent
A sequencer is a device that can produce rythmic loops programmed by the user. The loop is devided into 8 equally spaced steps and each step can be activated in order to produce a sound. The sequancer will go trough the loop and will play a sound on the activated steps. The loop period and the sound frequency can be set by the user.

%\newline
\noindent
The sequencer hardware is implemented in Verilog and uses the Picoversat SoC as the basic processing unit. Refer to the Picoversat manual for more information. 
This sequencer implementation is meant to be used on the Basys 2 FPGA by Digilent, for that, custom-made peripherals are developed in order to use the board's features (e.g.LEDs, Switches, etc...).
