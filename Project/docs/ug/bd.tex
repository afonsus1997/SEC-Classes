
\section{Sequencer operation}



In normal operation, the sequencer will loop through all the steps in a cyclic fashion, the current step can be seen by the indicated LED. In order to to create a rythm, the user can activate the desired steps by activating the correspondant switch. The activated steps are indicated by their correspondant LEDs being on.  

The sound output of the signal is a DC biased square wave outputted by an IO pin of the Basys 2 Board, this should be connected to and external powered speaker in order to the sound to be apmplified and filtered.


Figure~\ref{fig:bdbasys2} shows the Basys 2 peripherals that the user can use in order to interact with the sequencer.

\begin{figure}[!h]
  \centerline{\includegraphics[scale=1]{bdboard.png}}
  \vspace{0cm}\caption{Basys 2 board peripherals}
  \label{fig:bdbasys2}
\end{figure}

\section{Implementation}


Since the sequencer depends on multiple time-dependant routines and there is no trivial way to deal with this on the picoversat controller (because of the lack of interrupts), the main logic is divided into two routines: one for the main sequencer loop, implemented as a standalone peripheral, the ''Sequencer loop controller'', and other for the reading and debouncing of the pushbuttons and for keeping track of the frequency and loop values.


\noindent All the peripherals and modules are interconnected as described in the following picture:

\begin{figure}[!htbp]
  \centerline{\includegraphics[scale=0.5]{periphmoddiag.pdf}}
  \vspace{0cm}\caption{Basys 2 board peripherals}
  \label{fig:periphmoddiag}
\end{figure}




\noindent The sequencer loop controller is used to generate the sequencer loop. The loop and note frequency can be set by using the \textit{freq} input and by selecting the according selector signal.
The Sequencer loop controller will output a square wave corresponding to the loop ouput. The led outputs are directly connected to the LED driver peripheral and send information about the current note. 

\noindent The sequencer loop controller operation is described by the fluxogram on Picture \ref{fig:fluxloop}.


\begin{figure}[!htbp]
    \centerline{\includegraphics[scale=0.5]{SequencerLoopcontroler.png}}
    \vspace{0cm}\caption{PicoVersat SoC with two peripherals}
    \label{fig:periphs}
\end{figure}

% Please add the following required packages to your document preamble:
% \usepackage{booktabs}
\begin{table}[!htbp]
    \centering
    \caption{Sequencer Loop Controller Inputs}
    \label{tab:slcIn}
    \begin{tabular}{@{}lcl@{}}
    \toprule
    Name      & \multicolumn{1}{l}{\#bits} & Description                                    \\ \midrule
    freq & 8                          & Loop Period / Note Frequency                                    \\
    sel\_loop & 1                          & Loop period select signal (address decoder)    \\
    sel\_snd  & 1                          & Note frequency select signal (address decoder) \\ \bottomrule
    \end{tabular}
    \end{table}

    % Please add the following required packages to your document preamble:
% \usepackage{booktabs}
\begin{table}[!htbp]
    \centering
    \caption{Sequencer Loop Controller Outputs}
    \label{tab:slcOut}
    \begin{tabular}{@{}lcl@{}}
    \toprule
    Name          & \multicolumn{1}{l}{\#bits} & Description             \\ \midrule
    snd\_out      & 1                          & Audio Output            \\
    seq\_led\_out & 8                          & Current note led output \\ \bottomrule
    \end{tabular}
    \end{table}

\begin{figure}[!htbp]
  \centerline{\includegraphics[scale=0.4]{Sequencer_loop_controller_flux.pdf}}
  \vspace{0cm}\caption{Sequencer Loop controler fluxogram.}
  \label{fig:bd}
\end{figure}



\begin{figure}[htbp]
  \centerline{\includegraphics[scale=0.4]{picoversat_flux.pdf}}
  \vspace{0cm}\caption{Picoversat code fluxogram.}
  \label{fig:fluxloop}
\end{figure}

\newpage

\subsection{Block Diagram}


\begin{figure}[!h]
    \centerline{\includegraphics[scale=0.5]{bd.png}}
    \vspace{0cm}\caption{Block Diagram}
    \label{fig:bd}
\end{figure}




\begin{figure}[!htbp]
    \centerline{\includegraphics[scale=0.5]{periphs.png}}
    \vspace{0cm}\caption{PicoVersat SoC with two peripherals}
    \label{fig:periphs}
\end{figure}